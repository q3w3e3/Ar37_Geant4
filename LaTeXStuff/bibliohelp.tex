\chapter{Citations}

You can cite sources in two ways.  First, you can use the citation as
a noun: \textcite[Chapter~8.8.4]{aho-compilers} briefly discuss the
use of graph coloring in the register allocation phase of a compiler.
In this case, use the \string\textcite\ command.  Second, you can use
a citation as a parenthetical, which is not read aloud: Graph coloring
is one possibile way to allocate
registers \parencite[Chapter~8.8.4]{aho-compilers}.  Use the
\string\parencite\ command for this.

Both commands (\string\textcite\ and \string\parencite) take three
parameters, two of which are optional.  The first (optional) parameter
is a pre-note, the second (optional) parameter is a post-note, and the
third (mandatory) parameter is the citation
key \parencite[see][Section~3.7]{biblatex-manual}.  The citation in
the preceding sentence was made using the following command:

\begingroup\footnotesize
\begin{verbatim}
\parencite[see][Section~3.7]{biblatex-manual}
\end{verbatim}
\endgroup

If you give these commands just one optional argument (that is, one
enclosed in square brackets), it will be interpreted as a post-note.
If you want to give only a pre-note, leave the post-note empty
\parencite[see][]{biblatex-manual}:

\begingroup\footnotesize
\begin{verbatim}
\parencite[see][]{biblatex-manual}
\end{verbatim}
\endgroup

It is also possible to cite multiple sources in the same citation
%
\parencites%
  [see][Section~3.7]{biblatex-manual}%
  [regarding citations in general, see also][Section~5.3.2]%
    {biblatex-chicago-manual}%
\relax.
%
Use the command  \string\parencites\
for this.  For each citation, give it the same parameters as you would give
a single \string\parencite\
command.  It is good practice (but often not necessary) to end the command
in a \string\relax, so that no surprises ensue.

\begingroup\footnotesize
\begin{verbatim}
\parencites%
  [see][Section~3.7]{biblatex-manual}%
  [regarding citations in general, see also][Section~5.3.2]%
    {biblatex-chicago-manual}%
\relax.
\end{verbatim}
\endgroup

If you break the command into multiple lines, use the comment sign
to end each line, to prevent spurious spaces.

\section{The bibliography database}

You should add all the sources you want to cite in a separate
bibliography database written on the \textsc{Bib\TeX} format.  You can
use many bibliographical tools in creating and maintaining it, but it
is perfectly possible to write it by hand.  The name of your
bibliography database must be given as an argument to the
\string\addbibresource\ command.

The database in \textsc{Bib\TeX} format is a text file following
special formatting rules.  It consists of records, each of which
starts with an @~sign, which is then followed by the type of the
record.  The rest of the record goes inside curly braces.  For example,
the compilers book cited earlier \parencite{aho-compilers} can be
represented as the following record:

\begingroup\footnotesize
\begin{verbatim}
@Book{aho-compilers,
  author =       {Alfred V. Aho and Monica S. Lam and Ravi Sethi and
                  Jeffrey D. Ullman},
  title =        {Compilers},
  subtitle =     {Principles, Techniques, \& Tools},
  publisher =    {Pearson Addison Wesley},
  year =         2007,
  address =      {Boston},
  edition =      2
}
\end{verbatim}
\endgroup%

The type of this record is ``book''.  The first word inside the curly
braces is the citation key, which is used in the \string\textcite\ and
\string\parencite\ commands.  It is followed by a comma and a set of
named fields like ``author'', ``title'', ``subtitle'' and
``publisher''.  The content of the field is written inside curly
braces, although numerical data can be written without them.

The names of the authors are written mainly in the conventional way.
An alternative is to invert it, giving the surname first, followed by
a comma and the first name (``Aho, Alfred V.''), and in some special
cases this is mandatory.\footnote{For example, if the author has a
  double surname without a hyphen separating them; as one example, the
  name of Simon Peyton Jones should be written in the database as
  ``Peyton Jones, Simon''.}  If there are multiple authors, their
names must be separated by an ``and''.  If you do not list all
authors, put ``and others'' after the last listed name.

If the author of some source is an organization, its name must be written
within another set of curly braces \parencite[eg.][]{unicode620}:

\begingroup\footnotesize
\begin{verbatim}
@Book{unicode620,
  author =       {{Unicode Consortium}},
  title =        {The Unicode Standard, Version 6.2.0},
  year =         {2012},
  url =          {http://www.unicode.org/versions/Unicode6.2.0/},
  urldate =      {2013-01-29}
}
\end{verbatim}
\endgroup

If a source, for some reson, has no named author, leave the ``author''
field out ntirely.  In that case, the citation uses the source's
title \parencite[eg.][]{presidential-novel}:

\begingroup\footnotesize
\begin{verbatim}
@Book{presidential-novel,
  title =        {O},
  subtitle =     {A Presidential Novel},
  publisher =    {Simon \& Schuster},
  year =         {2011},
}
\end{verbatim}
\endgroup

A journal article \parencite[eg.][]{strachey-fundamentals} is given a
record like the following:

\begingroup\footnotesize
\begin{verbatim}
@Article{strachey-fundamentals,
  author =       {Christopher Strachey},
  title =        {Fundamental Concepts in Programming Languages},
  journal =      {Higher-Order and Symbolic Computation},
  year =         2000,
  volume =       13,
  number =       {1--2},
  pages =        {11--49},
  doi =          {10.1023/A:1010000313106}
}
\end{verbatim}
\endgroup

Note especially the field ``doi'', in which you can write the Digital
Object Idenifier (DOI) of the article.  It is usually a better choice
than any URL, as the DOI is a permanent identifier for the article.
Most DOIs are also convertible to URLs by prepending
\url{http://dx.doi.org/}.

If the DOI of an online source is not known (or there is none at all),
you can use the ``url'' field.  In that case, you should also give the
date on which you read the source, in the field ``urldate'' (using the
international standard format YYYY--MM--DD).  You should choose the
address with great care, so that it is as precise as possible and
remains valid as long as possible.  If the page has a specially
indicated permanent link (or permalink), use it.

When citing a WWW page that is not a book or an article or any other
formal publication, you can use the ``online'' record
type \parencite[eg.][]{debian-social-contract}:

\begingroup\footnotesize
\begin{verbatim}
@Online{debian-social-contract,
  title =        {Debian Social Contract},
  year =         {2004},
  url =          {http://www.debian.org/social_contract.en.html},
  urldate =      {2013-01-29}
}
\end{verbatim}
\endgroup

Some sources are edited collections of independent articles.  In that
case, you should generally cite a specific article in
it \parencite[eg.][]{prechelt-credibility} instead of the full
collection.  Even then, you should add both the collection and the
cited article as their own records, and use a ``crossref'' field in
the article record to refer to the collection:\footnote{It is
  permissible to combine the article and the collection into one
  InCollection record, for example if one cites only one article in
  the collection.  In that case, the title of the collection goes in a
  ``booktitle'' field, and no ``crossref'' field is used.}

\begingroup\footnotesize
\begin{verbatim}
@Collection{making-software,
  editor =       {Andy Oram and Greg Wilson},
  title =        {Making Software},
  subtitle =     {What Really Works, and Why We Believe It},
  publisher =    {O'Reilly},
  year =         2011
}
@InCollection{prechelt-credibility,
  author =       {Lutz Prechelt and Marian Petre},
  title =        {Credibility, or Why Should I Insist on Being
                  Convinced},
  crossref =     {making-software},
  pages =        {17--34}
}
\end{verbatim}
\endgroup

Note that a collection has an ``editor'' instead of an ``author''.

For more information about the structure of a bibliography databasem
see the \textsc{Bib\TeX} manual \parencite{bibtexing},
the \textsc{Bib\LaTeX} manual \parencite[Section~2]{biblatex-manual},
and the \textsc{Bib\LaTeX}-Chicago manual
\parencite[Sections 5.1--5.2]{biblatex-chicago-manual}.  There are
also more examples in the source code of this document.


\section{The bibliography}

The bibliography database is converted into the bibliography by using
the utility program {biber}.  It is fairly new, and is often missing
from machines whose \TeX\ installation is not up to date.  Of the
ssh-accessible Linux servers of the University, only charra.it.jyu.fi
has it at this time.  It is installable in Ubuntu since version~12.10
(Quantal Quetzal) and in Debian since version~7 (Wheezy).  For
Windows, use the 32-bit Mik\TeX\ package
miktex-biber-bin.\footnote{Last I looked, there was no 64-bit package
  of biber for Mik\TeX.}

On the command line, biber is simple to use.  Once \LaTeX (or
pdf\LaTeX) has been run once, invoke biber with the document name
(without the .tex part) as its argument.  After that, run \LaTeX\ (or
pdf\LaTeX) at least once, until the latest run does not request
another run.  For example:

\begingroup\footnotesize
\begin{verbatim}
$ pdflatex malliopas
[...]
Package biblatex Warning: Please (re)run Biber on the file:
(biblatex)                malliopas
(biblatex)                and rerun LaTeX afterwards.
[..]
Output written on malliopas.pdf (18 pages, 96855 bytes).
Transcript written on malliopas.log.
$ biber malliopas
INFO - This is Biber 0.9.9
[...]
INFO - Output to malliopas.bbl
$ pdflatex malliopas
[...]
LaTeX Warning: Label(s) may have changed. Rerun to get cross-references right.
[...]
Output written on malliopas.pdf (21 pages, 107373 bytes).
Transcript written on malliopas.log.
$ pdflatex malliopas
[...]
Output written on malliopas.pdf (21 pages, 107509 bytes).
Transcript written on malliopas.log.
\end{verbatim}
\endgroup
